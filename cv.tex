\documentclass[a4paper,10pt]{article}

%A Few Useful Packages
\usepackage{marvosym}
\usepackage{fontspec} %for loading fonts
\usepackage{xunicode,xltxtra,url,parskip} 	%other packages for formatting
\RequirePackage{color,graphicx}
\usepackage[usenames,dvipsnames]{xcolor}
\usepackage[big]{layaureo} 				%better formatting of the A4 page
\usepackage[misc]{ifsym}
\usepackage{geometry}
\usepackage{fontawesome}
\geometry{
 a4paper,
 total={210mm,297mm},
 left=15mm,
 right=15mm,
 top=10mm,
 bottom=10mm,
}
\usepackage{enumitem}
\setlist{leftmargin=4mm}

% an alternative to Layaureo can be ** \usepackage{fullpage} **
\usepackage{supertabular} 				%for Grades
\usepackage{titlesec}					%custom \section
\usepackage{longtable}
\usepackage{tabularx}
%Setup hyperref package, and colours for links
\usepackage{hyperref}
\definecolor{linkcolour}{rgb}{0,0.2,0.6}
\hypersetup{colorlinks,breaklinks,urlcolor=linkcolour, linkcolor=linkcolour}

%FONTS
\defaultfontfeatures{Mapping=tex-text}
%\setmainfont[SmallCapsFont = Fontin SmallCaps]{Fontin}
%%% modified for Karol Kozioł for ShareLaTeX use
\setmainfont[
SmallCapsFont = Fontin-SmallCaps.otf,
BoldFont = Fontin-Bold.otf,
ItalicFont = Fontin-Italic.otf
]
{Fontin.otf}
%%%

%CV Sections inspired by: 
%http://stefano.italians.nl/archives/26
\titleformat{\section}{\Large\scshape\raggedright}{}{0em}{}[\titlerule]
\titlespacing{\section}{0pt}{3pt}{3pt}
%Tweak a bit the top margin
%\addtolength{\voffset}{-1.3cm}

%Italian hyphenation for the word: ''corporations''
\hyphenation{im-pre-se}

% fake table item
\newcommand{\tabitem}{~~\llap{\textbullet}~~}
%-------------WATERMARK TEST [**not part of a CV**]---------------
\usepackage[absolute]{textpos}

\setlength{\TPHorizModule}{30mm}
\setlength{\TPVertModule}{\TPHorizModule}
\textblockorigin{2mm}{0.65\paperheight}
\setlength{\parindent}{0pt}

%--------------------BEGIN DOCUMENT----------------------
\begin{document}

%WATERMARK TEST [**not part of a CV**]---------------
%\font\wm=''Baskerville:color=787878'' at 8pt
%\font\wmweb=''Baskerville:color=FF1493'' at 8pt
%{\wm 
%	\begin{textblock}{1}(0,0)
%		\rotatebox{-90}{\parbox{500mm}{
%			Typeset by Alessandro Plasmati with \XeTeX\  \today\ for 
%			{\wmweb \href{http://www.aleplasmati.comuv.com}{aleplasmati.comuv.com}}
%		}
%	}
%	\end{textblock}
%}

\pagestyle{empty} % non-numbered pages

% \font\fb=''[cmr10]'' %for use with \LaTeX command

%--------------------TITLE-------------
\par{\centering
		{\hspace{-2mm}\Huge \textsc{Archit Sharma}
	}\\\large \textsc{Senior Undergraduate, IIT Kanpur}\\\normalsize \Letter\hspace{1mm}\href{mailto:architsh@iitk.ac.in}{architsh@iitk.ac.in} | \Mundus\hspace{1mm}\href{https://architsharma97.github.io/}{ architsharma97.github.io} | \faGithub \hspace{0mm} \href{https://github.com/architsharma97/}{architsharma97} \par}
	
%--------------------SECTIONS-----------------------------------
%Section: Education
\section{Education}
\begin{tabular}{rlrl}	
\textsc{2018} & \large Bachelor of Technology, \textsc{IIT Kanpur} & \textsc{2014} & \large DAV Public School, Amritsar \\
Expected & {Major}: \textit{Electrical Engineering} & & AISSCE, XII Board: 95/100\\
         & {Minor}: \textit{Artificial Intelligence and Linguistic Theory}& & AISSE, X Board: 10/10\\ 
         & {CPI}: \textsc{9.8/10} \\
         & \\
% \textsc{2014} & \large DAV Public School, Amritsar\\
%               & AISSCE, XII Board: 95/100\\
%               & AISSE, X Board: 10/10\\
\end{tabular}
%Section: Achievements
\section{Achievements}
\begin{tabular}{rl}
2017 & \textsc{Departmental Rank 2} out of 140 undergraduates in Electrical
Engineering, IIT Kanpur.\\
2017 & Awarded \textsc{Sri Singhasan Singh Scholarship} for highest CPI in Electrical Engineering, IIT Kanpur.\\
2017 & Awarded $A^*$ for exceptional performance in seven courses.\\
2017 & Shortlisted for the  prestigious \textsc{Rhodes Scholarship} by IIT Kanpur.\\
2017 & Selected for the MITACS Globalink Research Internship. \\
2016 & Awarded \textsc{Academic Excellence Award} by IIT Kanpur for the academic year 2015-16.\\
2016 & Secured 63 rank in online round for \textsc{ACM ICPC India Regionals}.\\
2016 & Selected for \textsc{Texas A\&M-IITK Summer Research Internship Program}, only \textsc{sophomore} to accomplish this.\\
2015 & Awarded \textsc{Academic Excellence Award} by IIT Kanpur for the academic year 2014-15.\\
2014 & Secured \textsc{All India Rank 376} in \textsc{JEE Advanced} among 150,000 students.\\
% 2014 & Secured 730 rank in \textsc{JEE Mains} among 1.2 million students. \\
2012 & Secured \textsc{first position} in IX-X category in \textsc{NASA Ames Space Settlement Design Contest} worldwide.\\
2010 & Awarded \textsc{National Talent Search Scholarship} (NTSE) by Govt. of India.

\end{tabular}
%Section: Work Experience
\section{Work Experience}
\centering
\begin{longtable}{r|p{15cm}}
% MILA
\textsc{May-Aug 2017} & \large \textsc{Research Intern} at \textsc{Montreal Institute of Learning Algorithms (MILA)}\\
& \textit{Dr. Yoshua Bengio, Professor, University of Montreal} | \href{https://github.com/architsharma97/MNIST}{Github}
\begin{itemize}
\item Investigated and evaluated performance of different gradient estimators (REINFORCE/Policy Gradients, Straight Through and Gumbel-Softmax) for computational graphs with discrete latent variables in a half and half MNIST generation problem.
\item Based on the idea of \textsc{synthetic gradients}, evaluated the benefits and limitations of a novel gradient estimator for discrete latent variables, with potential use in reinforcement learning and GANs for discrete data (such as language modelling) among others. 
\vspace*{-\baselineskip}
\end{itemize}
\\
\multicolumn{2}{c}{}\\
% Texas A&M
\textsc{May-Jul 2016} & \large \textsc{Research Intern} at \textsc{Texas A\&M University, USA}\\
& \textit{Dr. Srinivas Shakkottai, Associate Professor, ECE Department}
\begin{itemize}
 \item Worked on analyzing privacy of the user in DSRC enabled cars, which allows vehicle to vehicle/infrastructure communication.
 \item Broke down 229 GB \textsc{BSM dataset} from Ann Arbor, Michigan in 6 hours by processing on 20 cores of IBM's Ada Cluster, reducing from expected computation time of 4 days.
 \item Successfully demonstrated the lack of privacy in Random ID Switching protocol by reconstructing car routes with 98.37\% accuracy. Other driving behaviour based protocols under test. \vspace*{-\baselineskip}
\end{itemize}
\\
\multicolumn{2}{c}{}\\
% Hughes Systique
\textsc{May-Jul 2015} & \large \textsc{Android Application Development Intern}\\
& \textit{Hughes Systique, Gurgaon, India}
\begin{itemize}
 \item Worked on \textsc{GoSuraksheit}, an android based application for women safety. Integrated the new \textsc{Facebook API} for quick status updates containing custom message and user location in case of emergency. \vspace*{-\baselineskip}
\end{itemize}\\
\end{longtable}
% Section: Projects
\section{Projects}
\centering
\begin{longtable}{r|p{15cm}}
% Visual Dialog
\textsc{Jan-Apr 2017} & \large \textsc{Visual Dialog}\\
& \textit{Undergraduate Project under Dr. Vinay P. Namboodiri } | \href{https://github.com/architsharma97/VisualDialog}{Github}
\begin{itemize}
\item Implemented encoder-decoder framework based deep learning models for \href{https://visualdialog.org/}{\textsc{visual dialog}}, with the aim to answer sequence of questions based on an image.
\item Created a memory network based encoder for the input image, questions and the past conversation. The embeddings are fed into deep LSTM based decoder to generate the answers.
\item Also implemented a late fusion encoder. The performance for both the encoders was comparable to those reported in the original paper.\vspace*{-\baselineskip}
\end{itemize}\\
\multicolumn{2}{c}{}\\
% Dehazing GANs
\textsc{Mar-Apr 2017} & \large \textsc{GANs for Single Image Dehazing}\\
& \textit{Course Project for Visual Recognition under Dr. Vinay P. Namboodiri}
\begin{itemize}
\item Following the implementation of pix2pix for image-to-image translation, designed a GAN based architecture for image dehazing. This was the first work to try dehazing images in a GAN based setup. Tested the effect of reweigthing L1 and Adversarial loss. Stacking of generators was also explored as a part of this project.
\item Visually appealing results were obtained for 256x256 images, which were comparable to state-of-the-art techniques for dehazing images.\vspace*{-\baselineskip}
\end{itemize}\\
\multicolumn{2}{c}{}\\
% Video Summarization
\textsc{Sep-Nov 2016} & \large \textsc{Video Summarization}\\
& \textit{Course Project for Machine Learning under Dr. Piyush Rai } | \href{http://github.com/architsharma97/VideoSummarization/}{Github}
\begin{itemize}
\item Implemented VSUMM, a \textit{clustering based algorithm} using features extracted at frame level. Features tested upon include \textsc{color histograms} and \textsc{fc7 layer features} of \textsc{VGG16} architecture.
\item Implemented a \textit{deep learning model} using two opposite moving \textsc{LSTM} layers combined using a \textsc{1D convolution} layer.
\item Implemented a custom version of VGRAPH and SIFT based algorithm.\vspace*{-\baselineskip}
\end{itemize}\\
\multicolumn{2}{c}{}\\
% Visual Question Answering
Dec 2015 & \large \textsc{Visual Question Answering using Deep Neural Nets}\\
& \textit{under Dr. Vinay P. Namboodiri } | \href{https://github.com/architsharma97/VisualQA}{Github}
\begin{itemize}
\item Implemented a \textit{deep learning model} for VQA Dataset. Used the \textsc{VGG16 Architecture} to extract visual features. \textsc{word2vec} was used to convert questions to into word vectors, which are fed into a LSTM network. The results are combined using a fully connected MLP, giving a softmax distribution over 1000 most frequent answers.\vspace*{-\baselineskip}
\end{itemize}\\
\multicolumn{2}{c}{}\\
% Tagged Location Data Collection
\textsc{May 2016} & \large \textsc{Tagged Location Data Collection}\\
& \textit{Android Application developed at Texas A\&M University | } \href{https://github.com/architsharma97/LocationTagger}{Github}
\begin{itemize}
\item Minimalistic application to collect location data from users tagged with their mode of transport, developed from scratch.
\item Pushes locally written files, containing location data, using FTP to remote server in a multi-threaded environment.\vspace*{-\baselineskip}
\end{itemize}\\
\end{longtable}

% Section: Courses
\section{Coursework}
\centering
\begin{tabular}{rl|rl|rl}
Visual Recognition & A & Machine Learning & A & Probabilistic Machine Learning & \#\\
Fundamentals of Computing & A* & Data Structures and Algorithms & A & Image Processing & \#\\
Digital Signal Processing & A & Probability and Statistics & A* & Algorithms-II & \#\\
Introduction to Real Analysis & A & Partial Differential Equations & A\\
Linear Algebra and ODE & A & Digital Electronics & A\\
Priniciples of Communication & A & Control Systems & A*\\ \\

\multicolumn{4}{l}{\footnotesize A* $\equiv$ Outstanding}\\
\multicolumn{4}{l}{\footnotesize \# $\equiv$ Fall 2017}
\end{tabular}

% Section: Technical Skills
\section{Technical Skills}
\begin{tabular}{rp{15cm}}
Proficient & C++, C, Python, \LaTeX\\
Comfortable & JAVA, Shell (Bash), MATLAB\\
Tools & Theano, Git, Keras, NumPy, Scikit-Learn, Matplotlib\\
Operating Systems & Mac, Linux, Windows, Android
\end{tabular}

% Section: Miscellaneous
\section{Miscellaneous}
\begin{longtable}{p{16cm}}
\large \textsc{Software Corner Manager, Techkriti'16}\\
Handled logistics for software events in Techkriti, annual technical festival of IIT Kanpur. \textit{Managed a team of six} for conducting events and interacting with prospective participants. Conducted an onsite \textit{Appathon} with over 50 participants from colleges across India. Hosted the \textit{Algorithmic Programming Contest IOPC} with over 400 teams registered across India.\\ \\
\large \textsc{Competitive Programming}\\
\href{https://www.codechef.com/users/architsh}{\textsc{Codechef Long Challenge Rating:}} 8190.89. Over 80 problems solved on \href{http://www.spoj.com/users/architsh/}{SPOJ}.\\
Appeared in Amritapuri and Chennai regionals of ACM ICPC 2017 and Round 2 of Facebook Hackercup 2017\\ \\
\pagebreak
\large \textsc{Student Guide, Counselling Service}\\
\textit{Mentored} seven freshmen for their first year. Assisted in arranging a \textit{six day orientation} for incoming freshmen.\\ \\
\large \textsc{Secretary, Programming Club}\\
Organized lectures, workshops and contests for over 200 freshmen.
\\ \\
\large \textsc{Music}\\
Performed in \textit{Musical extravaganza} and \textit{Fresher's Day} as lead guitarist for Music Club.\\
Stood \textit{second} in eastern music competition in Galaxy'15, cultural event of IIT Kanpur.
\end{longtable}
\end{document}