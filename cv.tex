\documentclass[a4paper,10pt]{article}

%A Few Useful Packages
\usepackage{marvosym}
\usepackage{fontspec} %for loading fonts
\usepackage{xunicode,xltxtra,url,parskip} 	%other packages for formatting
\RequirePackage{color,graphicx}
\usepackage[usenames,dvipsnames]{xcolor}
\usepackage[big]{layaureo} 				%better formatting of the A4 page
\usepackage[misc]{ifsym}
\usepackage{geometry}
\usepackage{fontawesome}
\geometry{
 a4paper,
 total={210mm,297mm},
 left=15mm,
 right=15mm,
 top=10mm,
 bottom=10mm,
}
\usepackage{enumitem}
\setlist{leftmargin=4mm}

% an alternative to Layaureo can be ** \usepackage{fullpage} **
\usepackage{supertabular} 				%for Grades
\usepackage{titlesec}					%custom \section
\usepackage{longtable}
\usepackage{tabularx}
%Setup hyperref package, and colours for links
\usepackage{hyperref}
\definecolor{linkcolour}{rgb}{0,0.2,0.6}
\hypersetup{colorlinks,breaklinks,urlcolor=linkcolour, linkcolor=linkcolour}

%FONTS
\defaultfontfeatures{Mapping=tex-text}
%\setmainfont[SmallCapsFont = Fontin SmallCaps]{Fontin}
%%% modified for Karol Kozioł for ShareLaTeX use
\setmainfont[
SmallCapsFont = Fontin-SmallCaps.otf,
BoldFont = Fontin-Bold.otf,
ItalicFont = Fontin-Italic.otf
]
{Fontin.otf}
%%%

%CV Sections inspired by: 
%http://stefano.italians.nl/archives/26
\titleformat{\section}{\Large\scshape\raggedright}{}{0em}{}[\titlerule]
\titlespacing{\section}{0pt}{3pt}{3pt}
%Tweak a bit the top margin
%\addtolength{\voffset}{-1.3cm}

%Italian hyphenation for the word: ''corporations''
\hyphenation{im-pre-se}

% fake table item
\newcommand{\tabitem}{~~\llap{\textbullet}~~}
%-------------WATERMARK TEST [**not part of a CV**]---------------
\usepackage[absolute]{textpos}

\setlength{\TPHorizModule}{30mm}
\setlength{\TPVertModule}{\TPHorizModule}
\textblockorigin{2mm}{0.65\paperheight}
\setlength{\parindent}{0pt}

%--------------------BEGIN DOCUMENT----------------------
\begin{document}

%WATERMARK TEST [**not part of a CV**]---------------
%\font\wm=''Baskerville:color=787878'' at 8pt
%\font\wmweb=''Baskerville:color=FF1493'' at 8pt
%{\wm 
%	\begin{textblock}{1}(0,0)
%		\rotatebox{-90}{\parbox{500mm}{
%			Typeset by Alessandro Plasmati with \XeTeX\  \today\ for 
%			{\wmweb \href{http://www.aleplasmati.comuv.com}{aleplasmati.comuv.com}}
%		}
%	}
%	\end{textblock}
%}

\pagestyle{empty} % non-numbered pages

% \font\fb=''[cmr10]'' %for use with \LaTeX command

%--------------------TITLE-------------
\par{\centering
		{\hspace{-2mm}\Huge \textsc{Archit Sharma}
	}\\\large \textsc{Senior Undergraduate, IIT Kanpur}\\\normalsize \Letter\hspace{1mm}\href{mailto:architsh@iitk.ac.in}{architsh@iitk.ac.in} | \Mundus\hspace{1mm}\href{https://architsharma97.github.io/}{ architsharma97.github.io} | \faGithub \hspace{0mm} \href{https://github.com/architsharma97/}{architsharma97} \par}
	
%--------------------SECTIONS-----------------------------------
%Section: Education
\section{Education}
\begin{tabular}{rl}	
\textsc{2018} & \large Bachelor of Technology, \textbf{Indian Institute of Technology, Kanpur}\\
         & \textit{Major}: Electrical Engineering \\
         & \textit{Minors}: Machine Learning, Linguistic Theory\\
         & {CPI}: \textsc{9.9/10} \\
        %  & \\
% \textsc{2014} & \large DAV Public School, Amritsar\\
%               & AISSCE, XII Board: 95/100\\
%               & AISSE, X Board: 10/10\\
\end{tabular}

%Section: Research Experience
\section{Research Experience}
\centering
\begin{longtable}{r|p{15cm}}
% MILA
\textsc{May-Aug 2017} & \large \textsc{Synthetic Gradients across Discrete Latent Variables}\\
\faGithub \hspace{1mm}\href{https://github.com/architsharma97/MNIST}{github} & \textit{Research Intern at Montreal Institute of Learning Algorithms (MILA)}\\
\faFilePdfO \hspace{1mm} \href{https://docs.google.com/document/d/1eU1pLuHCfEBsj3-awdN1GR82oTQ15DK5jLm7ls4jlDM/edit#heading=h.rqi67q82now6}{summary} & \textit{Supervisor}: \textsc{Dr. Yoshua Bengio}

\begin{itemize}
\item Proposed a \textbf{novel estimator for gradients across discrete latent variables}, with potential use in Reinforcement Learning and GAN training (for structured data like language).
\item Formulated a synthetic gradient like auxiliary learner, with REINFORCE as the training signal, to produce \textbf{low-variance gradients} across discrete latent variables.
\item Compared the performance with other gradient estimators (REINFORCE, Straight Through and Gumbel-Softmax). The proposed estimator provided \textbf{faster but poorer convergence} compared to REINFORCE, possibly, because of the \textbf{non-stationary input} or the \textbf{mismatch between objectives of the auxiliary learner and the main network}.
\vspace*{-\baselineskip}
\end{itemize}
\\

\multicolumn{2}{c}{}\\
% Mixture of Bayesian SVMs
\textsc{Ongoing} & \large \textsc{Flexible Framework for Large-Margin Mixture-of-Experts}\\
\faFilePdfO \hspace{1mm}\href{https://architsharma97.github.io/resources/mbs.pdf}{presentation} & \textit{Undergraduate Project at IIT Kanpur}\\
\faFilePdfO \hspace{1mm}\href{https://architsharma97.github.io/resources/mbs_report.pdf}{report} & \textit{Supervisor}: \textsc{Dr. Piyush Rai}
\begin{itemize}
 \item  Leveraging recently developed \textbf{latent variable augmentation techniques}, we reformulate mixture-of-experts to provide efficient EM based learning algorithms.
 \item Our formulation provides \textbf{closed form EM updates} for a variety of gating networks, and even for \textbf{hierarchical mixture-of-experts}.
 \item Our models \textbf{achievs competitive results on various binary classification} datasets. \vspace*{-\baselineskip}
\end{itemize}\\

\multicolumn{2}{c}{}\\
% Texas A&M
\textsc{May-Jul 2016} & \large \textsc{Privacy Analysis of DSRC enabled Cars}\\
& \textit{Research Intern at Texas A\&M University}\\
& \textit{Supervisor}: \textsc{Dr. Srinivas Shakkottai}
\begin{itemize}
 \item Analyzed user privacy in DSRC enabled cars (vehicle to vehicle/infrastructure communication).
%  \item Broke down 229 GB \textsc{BSM dataset} from Ann Arbor, Michigan in 6 hours by processing on 20 cores of IBM's Ada Cluster, reducing from expected computation time of 4 days.
 \item Successfully \textbf{demonstrated the lack of privacy in Random ID Switching protocol} by reconstructing car routes with 98.37\% accuracy. \vspace*{-\baselineskip}
\end{itemize}\\
\end{longtable}

% Section: Selected Projects
\section{Selected Projects}
\centering
\begin{longtable}{r|p{15cm}}
\textsc{Ongoing} & \large \textsc{Improved Variational Inference using Real NVP}\\
\faFilePdfO \hspace{1mm} \href{https://architsharma97.github.io/resources/improvedvi.pdf}{report} & \textit{Course Project for Probabilistic Machine Learning under Dr. Piyush Rai}
\begin{itemize}
\item Proposed \textbf{Real NVP} as an alternate to Normalizing Flows in a VAE setup for generative modelling of images.
\item Compared Real NVP and Normalizing Flows on \textbf{Binarized MNIST, with promising initial results for the former}.
\item Further analysis with higher number of transformations, and on different datasets planned. \vspace*{-\baselineskip}
\end{itemize}\\
\multicolumn{2}{c}{}\\
% Visual Dialog
\textsc{Jan-Apr 2017} & \large \textsc{Visual Dialog}\\
\faGithub \hspace{1mm}\href{https://github.com/architsharma97/VisualDialog}{github} & \textit{Undergraduate Project under Dr. Vinay P. Namboodiri}
\begin{itemize}
\item Implemented \textbf{encoder-decoder framework} based deep learning models for \href{https://visualdialog.org/}{\textsc{visual dialog}}, with the aim to answer sequence of questions based on an image.
\item Created a \textbf{memory network based encoder} for the input image, questions and the past conversation and a \textbf{deep LSTM based decoder} to generate the answers.\vspace*{-\baselineskip}
% \item Also implemented a late fusion encoder. The performance for both the encoders was comparable to those reported in the original paper.
\end{itemize}\\
\multicolumn{2}{c}{}\\
% Dehazing GANs
\textsc{Mar-Apr 2017} & \large \textsc{GANs for Single Image Dehazing}\\
\faFilePdfO \hspace{1mm} \href{https://architsharma97.github.io/resources/dehazing_gans_poster.pdf}{poster} & \textit{Course Project for Visual Recognition under Dr. Vinay P. Namboodiri}
\begin{itemize}
\item Formulated a \textbf{deep architecture, along the lines of pix2pix, for single image-dehazing} with a weighted combination of GAN and L1 loss.\vspace*{-\baselineskip}
% \item Following the implementation of pix2pix for image-to-image translation, designed a GAN based architecture for image dehazing. This was the first work to try dehazing images in a GAN based setup. Tested the effect of reweighting L1 and Adversarial loss. \item Visually appealing results were obtained for 256x256 images, which were comparable to state-of-the-art techniques for dehazing images.
\end{itemize}\\
\multicolumn{2}{c}{}\\
% Video Summarization
\textsc{Sep-Nov 2016} & \large \textsc{Video Summarization}\\
\faGithub \hspace{1mm} \href{http://github.com/architsharma97/VideoSummarization/}{github} & \textit{Course Project for Machine Learning under Dr. Piyush Rai }
\begin{itemize}
\item Implemented and compared different video summarization techniques like VSUMM, VGRAPH on different features extracted at the frame level.\vspace*{-\baselineskip}
% \item Implemented VSUMM, a \textit{clustering based algorithm} using features extracted at frame level. Features tested upon include \textsc{color histograms} and \textsc{fc7 layer features} of \textsc{VGG16} architecture.
% \item Implemented a \textit{deep learning model} using two opposite moving \textsc{LSTM} layers combined using a \textsc{1D convolution} layer.
% \item Implemented a custom version of VGRAPH and SIFT based algorithm.\vspace*{-\baselineskip}
\end{itemize}\\
% \multicolumn{2}{c}{}\\
% % Visual Question Answering
% Dec 2015 & \large \textsc{Visual Question Answering using Deep Neural Nets}\\
% & \textit{under Dr. Vinay P. Namboodiri } | \href{https://github.com/architsharma97/VisualQA}{Github}
% \begin{itemize}
% \item Implemented a \textit{deep learning model} for VQA Dataset. Used the \textsc{VGG16 Architecture} to extract visual features. \textsc{word2vec} was used to convert questions to into word vectors, which are fed into a LSTM network. The results are combined using a fully connected MLP, giving a softmax distribution over 1000 most frequent answers.\vspace*{-\baselineskip}
% \end{itemize}\\
% \multicolumn{2}{c}{}\\
% % Tagged Location Data Collection
% \textsc{May 2016} & \large \textsc{Tagged Location Data Collection}\\
% & \textit{Android Application developed at Texas A\&M University | } \href{https://github.com/architsharma97/LocationTagger}{Github}
% \begin{itemize}
% \item Minimalistic application to collect location data from users tagged with their mode of transport, developed from scratch.
% \item Pushes locally written files, containing location data, using FTP to remote server in a multi-threaded environment.\vspace*{-\baselineskip}
% \end{itemize}\\

% \multicolumn{2}{c}{}\\
% % Hughes Systique
% \textsc{May-Jul 2015} & \large \textsc{Android Application Development Intern}\\
% & \textit{Hughes Systique, Gurgaon, India}
% \begin{itemize}
%  \item Worked on \textsc{GoSuraksheit}, an android based application for women safety. Integrated the new \textsc{Facebook API} for quick status updates containing custom message and user location in case of emergency. \vspace*{-\baselineskip}
% \end{itemize}\\
\end{longtable}

%Section: Achievements
\section{Achievements}
\begin{tabular}{rl}
2018 & \textbf{Departmental Rank 1} out of 140 undergraduates in Electrical
Engineering, IIT Kanpur.\\
2018 & Awarded \textbf{Academic Excellence Award} by IIT Kanpur for \textsc{consecutive academic years 2014-17}.\\
2018 & Awarded \textsc{Lalit Narain Das Memorial Scholarhip} by IIT Kanpur for the best B.Tech final year student in EE.\\
2018 & Awarded $A^*$ for \textbf{exceptional performance in 11 courses}.\\
2017 & Awarded \textsc{Sri Singhasan Singh Scholarship} for \textbf{highest CPI} in Electrical Engineering, IIT Kanpur.\\
% 2017 & Shortlisted for the  prestigious \textsc{Rhodes Scholarship} by IIT Kanpur.\\
% 2017 & Selected for the MITACS Globalink Research Internship. \\
% 2016 & \textbf{Secured 63 rank} in online round for \textsc{ACM ICPC India Regionals}.\\
2016 & Selected for \textbf{Texas A\&M-IITK Summer Research Internship Program}, only \textsc{sophomore} to accomplish this.\\
2014 & Secured \textbf{All India Rank 376} in \textsc{JEE Advanced} among 150,000 students.\\
% 2014 & Secured 730 rank in \textsc{JEE Mains} among 1.2 million students. \\
% 2012 & Secured \textsc{first position} in IX-X category in \textsc{NASA Ames Space Settlement Design Contest} worldwide.\\
2010 & Awarded \textbf{National Talent Search Scholarship} (NTSE) by Govt. of India.

\end{tabular}
% Section: Courses
\section{Relevant Coursework}
\centering
\begin{tabular}{rl|rl|rl}
Visual Recognition & A & Machine Learning & A & Probabilistic Machine Learning & A*\\
Fundamentals of Computing & A* & Data Structures and Algorithms & A & Image Processing & A*\\
Digital Signal Processing & A & Probability and Statistics & A* & Convex Optimization & A*\\
Introduction to Real Analysis & A & Partial Differential Equations & A & Statistical Learning Theory & A\\
Linear Algebra and ODE & A & Algorithms-II & A & Natural Language Processing & A\\ \\

\multicolumn{4}{l}{\footnotesize A* $\equiv$ Outstanding}\\
\end{tabular}

% Section: Technical Skills
\section{Technical Skills}
\begin{tabular}{rp{15cm}}
Proficient & C++, C, Python, \LaTeX\\
Comfortable & JAVA, Shell (Bash), MATLAB\\
Tools & Tensorflow, Theano, Git, NumPy, Scikit-Learn
\end{tabular}

% Section: Miscellaneous
\section{Miscellaneous}
\begin{longtable}{p{16cm}}
\large \textsc{Talks}: \normalsize Presented a \faFilePdfO  \hspace{1mm}\href{https://architsharma97.github.io/resources/mlrd_talk.pdf}{talk} on \textbf{Gradients for Discrete Latent Variables} on Machine Learning Research Day (MLRD) organized by SIGML, IIT Kanpur.\\ \\
\large \textsc{Project Mentor}: \normalsize Mentoring student projects in ``Topics in Probabilistic Modelling and Inference''. Chosen on the basis of exceptional performance in courses and relevant research experience.\\ \\
\large \textsc{Competitive Programming}: \normalsize
\href{https://www.codechef.com/users/architsh}{\textsc{Codechef Long Challenge Rating:}} 8190.89. Over 80 problems solved on \href{http://www.spoj.com/users/architsh/}{SPOJ}. Secured \textbf{63 rank} in online qualification round for \textbf{ACM ICPC Regionals 2017}, appearing thereafter in \textsc{Amritapuri and Chennai regionals}. Also, appeared in Round 2 of Facebook Hackercup 2017.\\ \\
\large \textsc{Standardized Scores}: \normalsize
\textbf{GRE}: 336/340, \textbf{TOEFL}: 115/120.\\ \\
\large \textsc{Software Corner Manager, Techkriti'16}: \normalsize
Handled logistics for software events in Techkriti, \textit{annual technical festival of IIT Kanpur}.\\ \\
\large \textsc{Android Development}: \normalsize
Integrated the Facebook API in \textit{GoSuraksheit}, a women safety application developed at \textsc{Hughes Systique}.Developed an android application to collect location and travel data at \textsc{Texas A\&M University}. \\ \\
\large \textsc{NASA Ames Space Settlement Design Contest, 2012}: \normalsize \textbf{Awarded first position} in IX-X category amongst \textbf{participants from over 10 countries} for designing a space settlement capable of hosting nearly 10,000 humans independently.\\ \\
\large \textsc{Student Guide, Counselling Service}: \normalsize
Mentored seven freshmen through their first year.\\ \\
\large \textsc{Secretary, Programming Club}: \normalsize
Organized lectures, workshops and contests for 200 freshmen.
\\ \\
\large \textsc{Music}: \normalsize
Played guitar in various competitions and events organized by Music Club, IIT Kanpur.
\end{longtable}
\end{document}